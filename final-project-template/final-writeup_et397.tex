\documentclass{article}

\usepackage{times}
\usepackage{graphicx}
\usepackage{subfigure}
\usepackage{natbib}
\usepackage{algorithm}
\usepackage{algorithmic}
\usepackage{amssymb}
\usepackage{lipsum}
\usepackage{todonotes}

\def\authnotes{1}

\newcounter{notectr}[section]
\newcommand{\thenote}{\thesubsection.\arabic{notectr}\refstepcounter{notectr}}

\newcommand{\fixme}[1]{\textcolor{red}{[FIXME: #1]}}
\newcommand{\note}[2]{$\ll$#1~\thenote: #2$\gg$}
\newcommand{\enote}[1]{\ifnum\authnotes=1 \textcolor{blue}{\note{EmilyT}{#1}}\fi}


\usepackage[accepted]{icml2017}
\icmltitlerunning{CS 287 Final Project Template}

\begin{document}

\twocolumn[
\icmltitle{Project Template}
\begin{icmlauthorlist}
  \icmlauthor{Emily Tseng}{}
% \icmlauthor{Yuntian Deng}{}
\end{icmlauthorlist}

\vskip 0.3in
]

\begin{abstract}
  \begin{itemize}
  \item This document describes the expected style, structure, and rough proportions for your final project write-up.
  \item While you are free to break from this structure, consider it a strong prior for our expectations of the final report.
  \item Length is a hard constraint. You are only allowed max \textbf{8 pages} in this format. While you can include supplementary material, it will not be factored into the grading process. It is your responsibility to convey the main contributions of the work in the length given.
  \end{itemize}


\end{abstract}

\section{Introduction}
\label{sec:introduction}

Example Structure:
\begin{itemize}
\item What is the problem of interest and what (high-level) are the current best methods for solving it?
\item How do you plan to improve/understand/modify this or related methods?
\item Preview your research process, list the contributions you made, and summarize your experimental findings.
\end{itemize}

In therapeutic and caregiving contexts, a patient’s felt sense of \textit{alliance} with a care provider can make or break their treatment. This is particularly important in remote caregiving contexts like online therapy, where the vast majority of the connection between the patient and the caregiver takes place entirely over text-chat. 

% Yet, to-date there have been few attempts to develop rigorous ways to measure how alliance manifests over a remote caregiving connection. What conversational cues or strategies, delivered over what modalities, best create a sense of alliance? How long does it take to build alliance, and how quickly can it dissolve? How is alliance re-established? How does alliance intersect with a patient’s sense of therapist fit, and what is the impact of switching therapists on a patient’s future alliance development?

In this work, we explored the use of neural representations of conversational dynamics in forecasting alliance between patients and therapists.

\section{Background}
Example Structure:
\begin{itemize}
\item What information does a non-expert need to know about the problem domain?
\item What data exists for this problem?
\item What are the challenges/opportunities inherent to the data? (High dimensional, sparse, missing data, noise, structure, discrete/continuous, etc?)
\end{itemize}


\section{Related Work}

Example Structure:
\begin{itemize}
\item What 3-5 papers have been published in this space?
\item How do these differ from your approach?
\item What data or methodologies do each of these works use?
\item How do you plan to compare to these methods?
\end{itemize}

\textbf{NLP and therapy.}

\textbf{Conversational forecasting.} Methodologically, our work is an adaptation of the Conversational Recurrent Architecture for ForecasTing (CRAFT), a framework integrating generative pre-training with a supervised fine-tuning model to achieve improved predictive ability on conversation-level attributes, e.g., whether an online conversation will derail into personal attacks \cite{Chang-Trouble:19}.

% \textbf{Pre-training and fine-tuning.}

\section{Dataset}

We consider a dataset of therapy transcripts and associated patient outcomes from Talkspace, an online therapy platform. Due to the highly sensitive nature of therapy, we put significant effort into respecting patients' privacy and autonomy. All represented patients gave informed consent for the use of their data in research, and transcripts were anonymized by Talkspace before they were handed to our research team. Our study protocol was approved by our institutional IRB.

In total, our dataset consists of 5.7M messages exchanged between patients and therapists, representing 11,233 patients' full courses of treatment. 1,906 therapists are represented, with an average of 9 patients per therapist (stdev \fixme{TODO}). \fixme{A smattering of more descriptive statistics: length of pt message, length of therapist message, number of pt / th messages per conversation, length of treatment, burstiness of treatment...}

Outcome annotations are provided in the form of patients' responses to surveys issued approximately every 3 weeks. In total, 13,742 WAI scores were provided by 6,702 patients. Patients provided an average of 2.1 WAI scores (range 1-24, stdev 2.2), and the overall distribution of scores skewed strongly towards the positive end of the spectrum (more strongly allied) \fixme{check and then insert figure}.


\section{Model}

Example Structure:

\begin{itemize}
\item What is the formal definition of your problem?
\item What is the precise mathematical model you are using to represent it? In almost all cases this will use the probabilistic language from class, e.g.
  \begin{equation}
  z \sim {\cal N}(0, \sigma^2)\label{eq:1}
\end{equation}
But it may also be a neural network, or a non-probabilistic loss,
\[ h_t \gets \mathrm{RNN}(x_{t}, h_{t-1} )\]

This is also a good place to reference a diagram such as Figure~\ref{fig:diagram}.

\item What are the parameters or latent variables of this model that you plan on estimating or inferring? Be explicit. How many are there? Which are you assuming are given? How do these relate to the original problem description?
\end{itemize}


\begin{figure}
  \centering
  \missingfigure[figheight=8cm]{}
  \caption{\label{fig:diagram} This is a good place to include a diagram showing how your model works. Examples include a graphical model or a neural network block diagram.}
\end{figure}

% Our model is an adaptation of the Conversational Recurrent Architecture for ForecasTing (CRAFT) \cite{Chang-Trouble:19}, which integrates a generative dialogue model and a supervised fine-tuning component. 

We define a conversation $C$ as a variable-length sequence of $n$ utterances, $C=\{u_1,...,u_n\}$. Utterances are variable-length sequences of tokens $w$, and thus $u_n=\{w_1,...,w_{M_n}\}$, where $M_n$ is the length in tokens of utterance $n$.

\textbf{Problem definition.} Given a therapy exchange $C=\{u_1,...,u_n\}$, we aim to predict $y_n$, the WAI score provided by the patient at utterance $u_n$.




\textbf{Generative component.} Following Chang et al. \citeyear{Chang-Trouble:19}, we adopted for our generative component the hierarchical recurrent encoder-decoder (HRED) architecture proposed in Sordoni et al. \citeyear{sordoni2015hierarchical} and Serban et al. \citeyear{serban2016building}. Built to model high-level conversational context, including temporal structure and dependencies between consecutive sequential inputs, HREDs are uniquely suited for conversational forecasting tasks. 

HREDs are comprised of three component recurrent neural networks (RNNs): an utterance encoder, a conversation encoder, and a decoder. First, the \textit{utterance encoder} generates for each utterance a semantic vector representation via its hidden state $h^{enc} \in \mathbb{R}^d_{enc}$, where $d_{enc}$ is the desired dimension. For each token $w_m$ in utterance $n$ of length $M$, the encoder updates its $h^{enc}$ like so:
\begin{equation}
  h^{enc}_m \gets f^{RNN}(w_{m}, h^{enc}_{m-1})
\end{equation}
The utterance encoder's hidden state at the last step, $h^{enc}_{M}$, in theory represents an embedding for the entire utterance. Following Serban et al. \citeyear{serban2016building}, $h^{enc}_{0}$ is initialized as the zero vector $\mathbf{0}$, and following Chang et al. \citeyear{Chang-Trouble:19}, we use the GRU \cite{cho2014learning} as our nonlinear gating function $f^{RNN}$.

Next, the \textit{conversation encoder} uses the hidden states from each consecutive comment in a sequence of length $N$ to produce an embedding $h^{con}_n$ for the conversation up to the utterance at that point ($u_N$):
\begin{equation}
  h^{con}_n \gets f^{RNN}(h^{enc}_{M_n}, h^{con}_{n-1})
\end{equation}
The conversation encoder also initializes its hidden state $h^{con}_0$ with the zero vector $\mathbf{0}$, and also uses the GRU as its nonlinearity. We denote the dimension of $h^{con}$ as $d_{con}$.

The \textit{decoder} uses the embedded conversational context $h^{con}_n$ to generate a response to utterance $n$. Following Sordoni et al. \citeyear{sordoni2015hierarchical}, it does this by first initializing its own hidden state $h^{dec} \in \mathbb{R}^{d_{dec}}$ using a nonlinear activation of $h^{con}_n$:
\begin{equation}
  h^{dec}_0 = \tanh(D h^{con}_n + b_0)
\end{equation}
Where $D \in \mathbb{R}^{d_{dec} \times d_{con}}$ projects the context embedding into decoder space, and $b_0 \in \mathbb{R}^{d_{dec}}$. 
The decoder then updates its own hidden state for each response token using the following recurrence:
\begin{equation}
  h^{dec}_{t} \gets f^{RNN}(w_{t-1}, h^{dec}_{t-1})
\end{equation}
The decoder then produces the next token in its response by producing a probability distribution over words from $h^{dec}_t$:
\begin{equation}
  w_t = f^{out}(h^{dec}_t)
\end{equation}
In our implementation, we use a simple feedforward network for $f^{out}$, including a softmax.


\textbf{Predictive component.}

\textbf{Parameters.}


\section{Inference (or Training)}

\begin{itemize}
\item How do you plan on training your parameters / inferring the
  states of your latent variables (MLE / MAP / Backprop / VI / EM / BP / ...)

\item What are the assumptions implicit in this technique? Is it an approximation or exact? If it is an approximation what bound does it optimize?

\item What is the explicit method / algorithm that you derive for learning these parameters?
\end{itemize}


\begin{algorithm}
  \begin{algorithmic}
    \STATE{\lipsum[1]}
  \end{algorithmic}
  \caption{Your Pseudocode}
\end{algorithm}




\section{Methods}

\begin{itemize}
\item What are the exact details of the dataset that you used? (Number of data points / standard or non-standard / synthetic or real / exact form of the data)

\item What are the exact details of the features you computed?


\item How did you train or run inference? (Optimization method / hyperparameter settings / amount of time ran / what did you implement versus borrow / how were baselines computed).

\item What are the exact details of the metric used?
\end{itemize}


\section{Results}

\begin{itemize}
\item What were the results comparing previous work / baseline systems / your systems on the main task?
\item What were the secondary results comparing the variants of your system?
\item This section should be fact based and relatively dry. What happened, what was significant?
\end{itemize}

\begin{table*}
  \centering
  \missingfigure{}
  \caption{This is usually a table. Tables with numbers are generally easier to read than graphs, so prefer when possible.}
  \label{fig:mainres}
\end{table*}


\begin{table}
  \centering
  \missingfigure[figheight=5cm]{}
  \caption{Secondary table or figure in results section.}
  \label{fig:mainres}
\end{table}


\section{Discussion}

\begin{itemize}
\item What conclusions can you draw from the results section?
\item Is there further analysis you can do into the results of the system? Here is a good place to include visualizations, graphs, qualitative analysis of your results.

\item What questions remain open? What did you think might work, but did not?
\end{itemize}


\begin{figure}
  \centering
  \missingfigure{}
  \missingfigure{}
  \missingfigure{}
  \caption{Visualizations of the internals of the system.}
\end{figure}

\section{Conclusion}

\begin{itemize}
\item What happened?
\item What next?
\end{itemize}


% \section*{Acknowledgements}

% \textbf{Do not} include acknowledgements in the initial version of
% the paper submitted for blind review.

% If a paper is accepted, the final camera-ready version can (and
% probably should) include acknowledgements. In this case, please
% place such acknowledgements in an unnumbered section at the
% end of the paper. Typically, this will include thanks to reviewers
% who gave useful comments, to colleagues who contributed to the ideas,
% and to funding agencies and corporate sponsors that provided financial
% support.


\bibliography{references}
\bibliographystyle{icml2017}

\end{document}
